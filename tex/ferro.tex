\section{Introduction}

\section{\pvdf{}} 

\section{Experimental Techniques}
\subsection{Contact Potential Difference and Surface Photovoltage}
Chapter \ref{chap:spv} deals exclusively with the contact potential difference (\cpd{}) and the surface photovoltage (\spv{}), so any theoretical or practical explanation will be skipped here.\\ For the scope of this Chapter, it is sufficient to mention that \cpd{} measures the work function of a sample compared to a known reference and that \spv{} can be an indicator of charges trapped at the surface and the overall quality of a chemical or physical surface modification. Previous work carried out at the Cahen group often focused on chemical surface modifications of silicon and it was shown for a range of chemicals that a (surface) dipole can influence the work function of the underlying silicon substrate. It is therefore assumed that successfully adsorbed \pvdf{} should influence the work function of silicon as measured by \cpd{} and that the direction of this influence be tune-able by changing the polarisation-direction of the ferroelectric polymer. It is furthermore assumed that a change in \spv{} may be taken as an indication of either less surface recombination or the presence of an electrical field at the surface.

\subsection{Determination of Excess Carrier Lifetimes}
In semiconductors, photogenerated excess charge carriers may recombine via different mechanisms. The effects of each of these mechanisms can be described by individual recombination rates, $S$ or by individual characteristic recombination `lifetimes', $\tau$. Generally, a distinction has to be made between recombination in the bulk as expressed by the bulk carrier lifetime \tbulk{} and recombination at surfaces as expressed by the surface lifetime \tsurf{}. These two lifetimes combine to an effective carrier lifetime \teff{} given by:
\begin{equation}
\label{efflifetime}
	\frac{1}{\teff} = \frac{1}{\tbulk} + \frac{1}{\tsurf} \, .
\end{equation}
For the bulk, three mechanism can be identified: radiative recombination (also known as band-to-band recombination), \trad{}, where carriers of opposite charge neutralise each other and emit a photon of corresponding energy; Shockley-Read-Hall recombination , \tsrh{}, where carriers recombine via traps within the bandgap and Auger recombination, \taug{}, where opposite carriers recombine and transfer their energy to a third carrier, which subsequently gives off its excess energy in the form of heat. In the presence of these mechanisms, \tbulk{}, is given by:
\begin{equation}
\label{bulklifetime}
	\frac{1}{\tbulk} = \frac{1}{\trad} + \frac{1}{\tsrh} + \frac{1}{\taug} \, .
\end{equation}
For indirect bandgap semiconductors such as silicon, \trad{} is typically large compared to the other times and may be neglected in Equation \eqref{bulklifetime}. Conversely, for (high quality) direct bandgap semiconductors such as GaAs, \trad{} is short and \tsrh{} is typically large and may often be neglected~\cite{gaaslifetime}. Because Auger recombination is a three-particle process, it is inherently less probable than its two-particle counterparts. However, the Auger recombination rate is a cubic function of excess carrier density and therefore becomes dominant at higher carrier concentrations~\cite{auger}. Surfaces are natural defects in the crystal structure and as such provide ample opportunity for carriers to recombine. It is more difficult to generally relate the surface recombination velocities \ssurf{} to a surface recombination lifetime which can be used in Equation \eqref{efflifetime}. For example, the two large surfaces of a wafer might have different recombination velocities and in very high quality semiconductors, even the comparatively small surfaces at the edges may influence the overall effective lifetime~\cite{edgerecom}.\\
One method to measure the effective lifetime of the excess carriers of a semiconductor is to convert its photoconductance into its excess carrier density via known mobility functions:
\begin{equation}
\label{carrierbalance}
	\Delta n = \frac{\Delta \sigma}{qW (\mu_n \mu_p)} \, ,
\end{equation}
where $\sigma$ is the conductance, $\Delta n$ denotes the excess carrier density, $W$ is the sample thickness, $q$ the charge and $\mu_n$ \& $\mu_p$ are the mobilities of negative and positive carriers respectively. The situation is complicated because the mobilities generally are functions of the doping densities and also of the excess carrier density~\cite{sintononline}. Therefore, some knowledge about the sample is required and modelling the characteristics of the sample is generally applied. The carrier concentration in Equation \eqref{carrierbalance} can be converted to lifetimes by solving the continuity equations to obtain:
\begin{equation}
\label{taugeneral}
	\teff (\Delta n) = \frac{\Delta n (t)}{G(t) - \frac{d\Delta n}{dt} } \, ,
\end{equation}
in which $G(t)$ is the photogeneration rate and $\Delta n$ is, again, the excess carrier density~\cite{nagel_lifetime}. Equation \eqref{taugeneral} is generally valid and was even derived for non-uniform photogeneration in the presence of significant surface recombination. Most measurements are realised by flashing a strong light on the sample and monitoring the time-evolution of its photoconductance via a radio-frequency bridge. When the duration of the flash is long, a quasi-steady state situation is achieved in which the excess carrier density can be seen as constant in time. Equation \eqref{taugeneral} simplifies to:
\begin{equation}
\label{qsspd}
	\teff (\Delta n) = \frac{\Delta n (t)}{G(t)} \, ,
\end{equation}
which is a good approximation for short lifetimes. A drawback of this method is that the generation rate must be known. Therefore the intensity of the flash has to be monitored, corrected for reflectivity and absorption by the sample. The opposite situation, a transient photoconductance decay measurement, is achieved then the flash is very brief and no illumination reaches the sample during the measurement. Then, the photogeneration rate can be assumed to be zero and Equation \eqref{taugeneral} can be solved to obtain:
\begin{equation}
\label{tpd}
	\teff (\Delta n) = - \frac{t}{\ln (\Delta n)} \, .
\end{equation}
This approach is only valid for relatively long lifetimes, but the intensity of the flash does not need not be monitored and reflectivity \& absorption of the sample do not need to be known, either. The combined quasi-steady-state, quasi-transient approach as described by Equation \eqref{taugeneral} yields the most accurate results, but is also the most involved. For this approach to work, the intensity of the flash must be measured in real-time and the absolute conductance of the sample must be known, as described by Sinton \emph{et~al.}~\cite{sinton1,sinton2}.\\
In all cases, if one is interested in the surface recombination velocities or lifetimes, \teff{} needs to be translated into \tbulk{}. Experimentally, this can most easily be achieved by using a high quality substrate in which $\tbulk \gg \teff$. If the latter is not the case, then there is an exact equation that can be used for transient measurements:
\begin{equation}
\label{exacttransient}
	\ssurf = \sqrt{D \left( \frac{1}{\teff} - \frac{1}{\tbulk} \right)} \tan \left( \frac{W}{2} \sqrt{D \left( \frac{1}{\teff} - \frac{1}{\tbulk} \right)} \right) \, ,
\end{equation}
where $D$ is the minority carrier diffusivity and the other symbols have their usual meaning~\cite{luke_lifetime}. A series of approximations to that equation exists and they are valid if $\teff \gg \frac{W^2}{\pi ^2 D}$ for transient measurements and if $\teff \gg \frac{W^2}{12 D}$ for quasi-steady state measurements~\cite{sproul_lifetime}. 
\subsection{Infrared Spectroscopy}
Infrared (\ir{}) spectroscopy is a vibrational spectroscopy which employs light with energies just below the visible spectrum, \ie{} with wavelengths longer than $\sim$ 800 nm. It is common to express the spectrum in terms of its frequency and typical units are reciprocal centimeters (\si{\per\centi\metre}), also called `wavenumbers' and to divide it into three broad regions: `near \ir{}' ($\sim$ \numrange{14e3}{4e3}\si{\per\centi\metre}), `mid \ir{}' ($\sim$ \numrange{4e3}{0.4e3}\si{\per\centi\metre}) and `far \ir{}' ($\sim$ \numrange{400}{10}\si{\per\centi\metre}), where each region approximately excites different types of vibrations ranging from harmonic to rotational-vibrational excitations. Vibrational modes are \ir{}-active if there is a change in dipole moment which enables interaction with the electric field of the radiation. Therefore \ir{} spectroscopy is very sensitive to the chemical environment of the sample under investigation and is especially frequently used in organic chemistry. Practically, instead of scanning each wavenumber individually, the sample is often illuminated by a wide range of radiation and decomposed by a Fourier transformation into its components, yielding Fourier Transform \ir{} (\ftir{}) spectroscopy which shortens the time of the measurement and allows detecting small quantities of and minute changes in a sample.\\
To study thin films that are either physically or chemically adsorbed to a substrate surface, it is useful exclude interaction of the substrate with the \ir{}-radiation and to maximise the interactions interactions that are of interest. One method that accomplishes  both these goals is Attenuated Total Reflectance (\atr{}) \ftir{} where the \ir{} beam is directed into a crystal of high refractive index and reflects internally from its surfaces. The thin film surface under investigation is pressed onto the crystal to form an intimate contact so that an evanescent \ir{} wave can project into it. There, part of the \ir{} spectrum is absorbed by the sample and the now somewhat reduces beam is returned into the crystal where it can be internally reflected to be projected into the thin film again multiple times, greatly attenuating the absorption of the thin film. Refer to Figure \ref{fig:atr} for a schematic of the \atr{} method. As can be deduced from Figure \ref{fig:atr}, the reflectance is given by
\begin{equation}
\label{atr-reflectance}
	R(\lambda)^N = (1-a_{\lambda} \cdot d_e)^N \, ,
\end{equation}
where $N$ is the number of reflections, $a_{\lambda}$ is the wavelength specific absorptivity of the thin film and $d_e$ is its effective thickness. One complication that needs to be mentioned is that \atr{} spectra are not linear with wavenumber CITE VILAN and therefore sometimes difficult to compare to transmittance spectra. In the context of this study, that complication is somewhat alleviated because we will mainly compare \atr{}-\ftir{} spectra relative to other other \atr{}-\ftir{} spectra to assess whether or not a phase transition has occurred in the thin film sample and will for the most part not need absolute quantification of the samples.
\subsection{Ellipsometry}
Generally, ellipsometry is an optical technique for investigating the dielectric properties of (layers of) thin films of material(s) on reflective surfaces. It is an indirect method insofar as that the information obtained from an ellipsometry measurement cannot directly be converted into the dielectric properties of the thin film under investigation. Therefore, the measured data must be compared to a model and the success of an ellipsometry measurement hinges on the quality of that model. In its most common form, ellipsometry measures the properties of light reflected from the surface of a sample. The set-up consists of a light source, a polariser, the sample, an analyser and a detector. Light with known, elliptical polarisation is employed, hence the name. This incident light can be decomposed into two components: the \emph{s} component where the electrical field of the radiation oscillates parallel to the sample and perpendicular to the plane of incidence and the \emph{p} component where the field oscillates parallel to the plane of incidence. Ellipsometry measures the complex reflectance $\rho$ of the sample which consists of an amplitude component $\Psi$ and a phase difference $\Delta$ according to
\begin{equation}
	\rho = \tan \left( \Psi \right) \exp \left(i \Delta ) \, .
\end{equation}
The complex nature of the reflectance as measured by ellipsometry is readily apparent from the equation above. Per wavelength, one pair of $\Psi$ \& $\Delta$ can be obtained. Therefore, to reach more optical parameters, a broad spectrum is often employed in ellipsometry.  
\subsection{Electron Microscopy}
\subsection{IV}
\subsection{Dynamic Light Scattering}
Dynamic Light Scattering (\dls{}) is most commonly used to analyse nanoparticles, specifically it can be used to determine the size of nanoparticles in solution. The optical set-up of a \dls{} experiment is shown in Figure \ref{fig:dls-setup}, a typical example of the the obtained optical signal is shown in Figure \ref{fig:dls-signal}. A laser illuminates the sample, some amount of light is scattered by particles in solution and its intensity as a function of time is determined by the detector at right angles to the path of the laser beam. The `noise' in the measured intensity is a direct consequence of the (Brownian) motion of nanoparticles in solution and will be used to extract particle size. As particles move into and out of the path of the laser, more or less light will be scattered toward the detector. Furthermore, a particle that moves about but stays within the path of the beam for a certain delay time $\tau$ will continue to scatter light. Therefore, the evolution of the `noise' in time can be understood in terms of an autocorrelation function. The mathematical treatment of the signal from a population of nanoparticles with different sizes is complicated and mathematically ill posed. To overcome this difficulty experimentally, restrictions on the size distribution have to be supplied. The following discussion will be restricted to a nanoparticle population of a single size so as not to exceed the scope of this project. For such a population, the autocorrelation function
\begin{equation}
	C(\tau) = \sum _{t=0}^T I(t)\cdot I(t + \tau) \, ,
\end{equation}
where $T$ is the total time of the experiment, $I(t)$ is the intensity at time $t$ and $I(t + \tau)$ is the intensity at time $t+\tau$, will also be given by a simple exponential decay:
\begin{equation}
	C(\tau) = \exp (-2 \Gamma \tau) + B \, ,
\end{equation}
where $B$ is a baseline intensity and where $\Gamma$ is obtained by a data-fit and is related to the (translational) diffusion coefficient $D_t$ by the scattering vector $q$ as:
\begin{equation}
	\Gamma = D_t \cdot q^2 = D_t \left( \frac{4 \pi n}{\lambda} \sin \left( \frac{\theta}{2} \right) \right) ^2 \, .
\end{equation}
In the last equation $n$ is the refractive index of the medium, $\lambda$ is the wavelength of scattered light and $\theta$ is the scattering angle. Finally, $D_t$ as determined from the data-fit can be used in the Stokes-Einstein relation for Brownian motion to find the apparent hydrodynamic diameter $D_h$ of the suspended particles:
\begin{equation}
\label{diameter}
	D_h = \frac{k_B T}{3\pi \eta D_t} \, .
\end{equation}
Here, $k_B$ is the Boltzmann constant, $T$ is the thermodynamic temperature and $\eta$ is the viscosity of the suspending medium. As can easily be seen in Equation \eqref{diameter}, the determined particle diameter is directly proportional to the absolute temperature $T$, however, the viscosity of the medium is usually also very sensitively dependent on temperature and a specific value for $\eta$ needs to be supplied for the measurement. For practical purposes, the viscosity of typical (mixtures of) solvents in a range of temperatures is internally tabulated by software, as are their refractive indices. Therefore, the user of a \dls{} apparatus supplies the type of solvent used, the machine monitors the temperature during the experiment and carries out the necessary calculations and data-fitting routines. It is important to note that, for a successful size determination via \dls{}, a solution of nanoparticles should be sufficiently dilute to avoid clusters of nanoparticles while at the same time being concentrated enough to allow for measurable scattering intensities. Furthermore, the Stokes-Einstein relation assumes rigid, spherical particles so the apparent diameter of a soft, ovoid particle may significantly differ from its true dimensions.

\section{Experimental Work \& Results}
\subsection{Sample Prep}
\subsubsection{Si-H}
\subsection{\pvdf{} poling, annealing}
\subsubsection{Nanoparticles}
\subsection{CPD SPV}
\subsubsection{Standard Experiments}
\subsubsection{Pole-Reverse-Pole}
\subsection{Lifetimes}
\subsection{Ellipsometry}
\subsection{\ir{}}
\subsection{IV}
\subsection{Microscopy}
\subsection{\dls{}}


\section{Discussion}