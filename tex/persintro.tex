\begin{personalintro}
\addchaptertocentry{\introname}
This Thesis will outline my attempts at enhancing solar cell efficiency with the help of a ferroelectric polymer, poly(vinylidene difluoride), and my work with the \McA{} Kelvin Probe and the Lakeshore cryogenic system. As is probably so often the case, my Master Project took longer than it should have and it was more difficult than it probably needed to be.\\
The research I am going to present here was carried out more than a year ago in Israel. I came back from Israel and had -- somehow -- given up. I did not think my \enhyphen{results} would be good enough to be able to call myself \degreename{} and things back home in the Netherlands swept me away as well. So I let the project sit and failed to stay in contact with the people I met in Israel. I did not offer an explanation to my supervisors, my behaviour was not that of a responsible adult, I was simply gone.\\
The birth of this project in Israel was complicated. I initially set out to work on a project involving gallium arsenide surface passivation and modification for ultra-hot carrier cells and had already spent some effort in understanding the material and in reading the necessary literature. Gallium arsenide and especially its solar cells are dear to me because of my experiences in the Applied Materials Science group at my home university and I was looking forward to the challenges of the surface passivation, GaAs is notorious for degrading in the blink of an eye.\\
But it should not be. The project was supposed to be in cooperation with a group in France and at some point during my stay, it transpired that their method would necessitate concentrated light intensities that no molecular surface modification could realistically survive.\\
It was my personal decision to join a project that had gone on for some time, but had so far failed to make meaningful progress. I wanted to work with solar cells! So I joined Dr. Ann Erickson in her efforts to make our cells work with a ferroelectric polymer. Shortly after, Ann left the group to emigrate back to the U.S. and I suddenly had the main responsibility for the project, a project I did not even fully understand at the time.\\
Meanwhile, I was working on a side project, which initially was not even supposed to be part of my Thesis research. But I failed to make progress on my main topic, so it was easier and more rewarding to put in more and more time for the side project. The \McA{} became my friend. It had its quirks, but I felt I was maybe getting somewhere, only my main topic suffered.\\
Toward the end of my stay in Israel, a conflict between Israelis and Palestinians broke out and while I adopted quite well, my family back in Germany did not see the reality on the streets, but that presented by the media, a much harsher situation than I actually found myself in. A lot of energy that should have been put in to carrying out the last experiments, to finally achieve some interesting results and maybe even to start working on my Thesis went into explaining that my life was not, in fact, in danger. Or at least not outrageously so.\\
So there were some external influences that did not help, but the main problem was mine, my own un-professionalism in not trying to finish what I started and my indecency in abandoning the trust that my supervisors had placed in me.\\
Therefore, I am more than merely thankful that both Professor Cahen and Doctor Meekes still gave me a chance to bring this project to completion when I finally had come around enough to face my mistakes and to somehow try to make them right. Thank you both.\\
\end{personalintro}