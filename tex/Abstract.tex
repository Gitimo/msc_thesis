%----------------------------------------------------------------------------------------
%	ABSTRACT PAGE
%----------------------------------------------------------------------------------------

\begin{abstract}
\addchaptertocentry{\abstractname} % Add the abstract to the table of contents

The project consists of two parts, dealing with applications of a ferroelectric polymer for use in solar cells and with setting up an experimental system capable of measuring temperature dependent surface photovoltage respectively.\\
The ultimate goal of the first part, namely to \emph{enhance} the power conversion efficiency of silicon/organic Schottky-type cells could not be attained. The correct deposition parameters for obtaining high quality layers of ferroelectric polymer could be identified, specifically cyclohexanone could be identified as the \enhyphen{best} solvent for the polymer, where thin layers ($\sim$\SI{1}{\nano\metre}) proved most practical for applications in solar cells. The poling procedure employed for the thin films proved to degrade the solar cells. Nanoparticles of the ferroelectric polymer were synthesised, but the material could not be crystallised in the ferroelectrically active phase and the particles were thus not useful for enhancing the characteristics of solar cells.\\
In the second part, successive tests proved that the experimental system was capable to reproduce results from established systems and a temperature dependent measurement of the surface photovoltage could be shown in a suitable model system could be shown. The precise nature of the results obtained from that model system remains unknown, but it never the less served as a proof of concept for the system.

\end{abstract}

